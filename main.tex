\documentclass[11pt,paper=a4,answers]{exam}
% Allow the usage of UTF-8 characters
\usepackage[utf8]{inputenc}
% Allow the usage of graphics (.png, .jpg)


\usepackage{graphicx}
\usepackage{caption}
\usepackage{amsmath}
\usepackage{amsfonts}
\usepackage{amssymb}
\usepackage{multicol}
\usepackage{color}
%\usepackage{stix}
\usepackage{circuitikz}
\setlength{\columnseprule}{1pt}
\setlength{\columnsep}{1cm}
\def\columnseprulecolor{\color{black}}
\usepackage{blind text}
\usepackage{tikz}
%\usepackage{minipage}
\usepackage{lscape}
\usetikzlibrary{angles,quotes}
\usepackage{siunitx}
\usepackage[pdftex]{pict2e}
% Start the document
\begin{document}


\title{EAPCET - 2024 - QUESTION BANK}
\author{UNIC ACADEMY Team}
\maketitle
\newpage

\section*{18 May 2024 Forenoon}
\noindent \\
\rule{\textwidth}{1.4pt}

\subsection*{Mathematics}
\begin{multicols}{2}
\begin{questions}



\question
If a function $f: \mathrm{R} \rightarrow \mathrm{R}$ is defined by $f(x)=x^3-x$, then $f$ is




\begin{flushright}
\small\textbf{AP EAPCET 2024}
\end{flushright}


\begin{choices}
\choice  one - one and onto
\choice one - one but not onto
\choice onto but not one - one
\choice neither one - one nor onto
\end{choices}


\question
If $f(x)=\sqrt{x}-1$ and $g\{f(x)\}=x+2 \sqrt{x}+1$ then $g(x)=$


\begin{flushright}
\small\textbf{AP EAPCET 2024}
\end{flushright}


\begin{choices}
\choice  $(x+2)^2$
\choice $(x-2)^2$
\choice $(\sqrt{x}+2)^2$
\choice $(\sqrt{x}-2)^2$
\end{choices}


\question
For all positive integers ' $n$ ' if \\ $3\left(5^{2 n+1}\right)+2^{3 n+1}$ is divisible by k, then the number of prime numbers less than or equal to $k$ is


\begin{flushright}
\small\textbf{AP EAPCET 2024}
\end{flushright}


\begin{choices}
\choice  $\displaystyle 17$ 
\choice $\displaystyle 6$ 
\choice $\displaystyle 7$ 
\choice $\displaystyle 8$ 
\end{choices}



\question
If $\alpha, \beta, \gamma$ are the roots of \\  $\left|\begin{array}{ccc}1-x & -2 & 1 \\ -2 & 4-x & -2 \\ 1 & -2 & 1-x\end{array}\right|=0$, then \\ $\alpha \beta+\beta \gamma+\gamma \alpha=$


\begin{flushright}
\small\textbf{AP EAPCET 2024}
\end{flushright}


\begin{choices}
\choice  $\displaystyle 6$ 
\choice $\displaystyle 8$ 
\choice $\displaystyle 0$ 
\choice $\displaystyle - 4$ 
\end{choices}



\question
If the determinant of a $3^{\text {rd }}$ order matrix A is K , then the sum of the determinants of the matrices $\left(A A^T\right)$ and $\left(A-A^T\right)$ is


\begin{flushright}
\small\textbf{AP EAPCET 2024}
\end{flushright}


\begin{choices}
\choice  $\displaystyle 2K$ 
\choice $\displaystyle 0$ 
\choice $\displaystyle K^2$ 
\choice $\displaystyle K$ 
\end{choices}




\question
While solving a system of linear equations $\mathrm{AX}=\mathrm{B}$ using Cramer's rule with the usual notation if $\Delta=\left|\begin{array}{ccc}1 & 1 & 1 \\ 2 & -1 & 2 \\ -1 & 1 & 5\end{array}\right| ; \Delta_1=\left|\begin{array}{ccc}5 & 1 & 1 \\ 4 & -1 & 2 \\ 11 & 1 & 5\end{array}\right|$ and $X=\left[\begin{array}{l}\alpha \\ 2 \\ \beta\end{array}\right]$, then $\alpha^2+\beta^2=$

\begin{flushright}
\small\textbf{AP EAPCET 2024}
\end{flushright}


\begin{choices}
\choice  $\displaystyle 9$ 
\choice $\displaystyle 13$ 
\choice $\displaystyle 5$ 
\choice $\displaystyle 25$ 
\end{choices}


\question
If real parts of $\sqrt{-5-12 i}, \sqrt{5+12 i}$ are positive values, the real part of $\sqrt{-8-6 i}$ is a negative value and $a+i b=\dfrac{\sqrt{-5-12 i}+\sqrt{5+12 i}}{\sqrt{-8-6 i}}$ then $2 a+b=$

\begin{flushright}
\small\textbf{AP EAPCET 2024}
\end{flushright}


\begin{choices}
\choice  $\displaystyle 3$ 
\choice $\displaystyle 2$ 
\choice $\displaystyle -3$ 
\choice $\displaystyle -2$ 
\end{choices}























\end{questions}
\end{multicols}

\begin{center}
\Large{Take the value of $\pi$ as 180${^\circ}$.}
\end{center}
%\end{document}

\newpage








\subsection*{Physics}
\begin{multicols}{2}
\begin{questions}
\question
If $\tan  \theta = \sqrt{3}$, then the value of $\sec \theta$ is
\begin{flushright}
\small\textbf{AP Poly CET 2023}
\end{flushright}


\begin{choices}
\choice $\displaystyle 2$ 
\choice $\displaystyle \frac{1}{2}$ 
\choice $\displaystyle \frac{\sqrt{3}}{2}$ 
\choice $\displaystyle \frac{2}{\sqrt{3}}$  
\end{choices}
\end{questions}

\end{multicols}

\begin{center}
\Large{Take the value of $\pi$ as 180${^\circ}$.}
\end{center}
%\end{document}

\newpage















\subsection*{Chemistry}
\begin{multicols}{2}
\begin{questions}
\question
If $\tan  \theta = \sqrt{3}$, then the value of $\sec \theta$ is
\begin{flushright}
\small\textbf{AP Poly CET 2023}
\end{flushright}


\begin{choices}
\choice $\displaystyle 2$ 
\choice $\displaystyle \frac{1}{2}$ 
\choice $\displaystyle \frac{\sqrt{3}}{2}$ 
\choice $\displaystyle \frac{2}{\sqrt{3}}$  
\end{choices}






\question If $\tan  \theta + \cot \theta = 5$, then the value of $\tan^2  \theta + \cot^2 \theta $ is
\begin{flushright}
\small\textbf{AP Poly CET 2023}
\end{flushright}

\begin{choices}
\choice $\displaystyle 1$ 
\choice $\displaystyle 7$ 
\choice $\displaystyle 23$ 
\choice $\displaystyle 25$  
\end{choices}
\question If $\displaystyle \sec  \theta = \frac{2}{\sqrt{3}}$, then $\cos \theta$ is
\begin{flushright}
\small\textbf{TS Poly CET 2023}
\end{flushright}


\begin{choices}
\choice $\displaystyle \frac{1}{\sqrt{2}}$ 
\choice $\displaystyle \frac{1}{\sqrt{3}}$ 
\choice $\displaystyle \frac{\sqrt{3}}{2}$ 
\choice $\displaystyle \frac{2}{\sqrt{3}}$   
\end{choices}


\columnbreak
\question What will be the value of  $\sin^2 \theta \left(1+\cot^2 \theta \right)$?

\begin{flushright}
\small\textbf{Bihar STET TGT 2023}
\end{flushright}


\begin{choices}
\choice 0
\choice 1
\choice $-1$
\choice   2
\end{choices}
\question The value of $\displaystyle \frac{\tan \alpha}{\sqrt{1+ \tan^2 \alpha} }  $ is
\begin{flushright}
\small\textbf{TS Poly CET 2022}
\end{flushright}


\begin{choices}
\choice $\displaystyle \cos \alpha$
\choice $\displaystyle \sin \alpha$
\choice $\displaystyle \textrm{cosec} \ \alpha$ 
\choice $\displaystyle \sec \alpha$
\end{choices}


\question If $\angle A=45^{\circ}$, $\angle B=60^{\circ}$, then $\sin A + \cos B$
\begin{flushright}
\small\textbf{TS Poly CET 2022}
\end{flushright}
\begin{choices}
\choice $\displaystyle \frac{2-\sqrt{2}}{2\sqrt{2}}$
\choice $\displaystyle \frac{2+\sqrt{2}}{{2}}$
\choice $\displaystyle \frac{2+\sqrt{2}}{\sqrt{2}}$
\choice $\displaystyle \frac{2+\sqrt{2}}{2\sqrt{2}}$
\end{choices}
\question 
If $\sin  \theta = \cos \theta \left(0^{\circ} < \theta < 90^{\circ}  \right)$, then $\tan \theta$ is
\begin{flushright}
\small\textbf{TS Poly CET 2022}
\end{flushright}


\begin{choices}
\choice $\displaystyle -1$ 
\choice $\displaystyle 4$ 
\choice $\displaystyle 2$ 
\choice $\displaystyle 1$  
\end{choices}
\columnbreak
\question If $\displaystyle \tan  \left(A-B \right) = \frac{1}{\sqrt{3}}$ and $\displaystyle \cos  \left(A \right) = \frac{1}{2}$, then $\angle B $ = ?
\begin{flushright}
\small\textbf{TS Poly CET 2022}
\end{flushright}

\begin{choices}
\choice $\displaystyle \frac{2 \pi }{3}$ 
\choice $\displaystyle \frac{\pi}{4}$
\choice $\displaystyle \frac{\pi}{6}$
 \choice $\displaystyle \frac{\pi}{3}$  
\end{choices}
\question If $\sec \theta + \tan  \theta = m$ and $\sec \theta - \tan  \theta = n$, then the value $nm$ is

\begin{choices}
\choice 0
\choice 1
\choice -1
\choice 0.5
\end{choices}
\question What is the value of $9 \cot^2 \theta -9\ \textrm{cosec}^2 \theta$


\begin{choices}
\choice 9
\choice $-9$
\choice 0
\choice 18
\end{choices}
\subsection*{Section B}
\question If $\displaystyle \sin \left(A-B \right) = \frac{1}{2}$ and $\displaystyle \cos \left( A+B \right) = \frac{1}{2}$ then $\angle A$, $\angle B$ = ?

\begin{flushright}
\small\textbf{TS Poly CET 2023}
\end{flushright}


\begin{choices}
\choice $\displaystyle 45^{\circ}, 15^{\circ}$ 
\choice $\displaystyle 15^{\circ}, 45^{\circ}$ 
\choice $\displaystyle 45^{\circ}, 30^{\circ}$ 
\choice $\displaystyle 30^{\circ}, 15^{\circ}$ 
\end{choices}

\columnbreak
\question If $\sin \theta + \cos \theta = \sqrt{3}$, then what is  $\tan \theta + \cot \theta$ equal to?
\begin{flushright}
\small\textbf{Bihar STET TGT 2023}
\end{flushright}


\begin{choices}
\choice $\displaystyle 1$ 
\choice $\displaystyle 0$  
\choice $\displaystyle \frac{1}{\sqrt{3}}$  
\choice $\displaystyle \frac{2}{\sqrt{3}}$  
\end{choices}


\question If $\sqrt{3} \tan  \theta = 1$, then the find the value of $\sin^2 \theta  - \cos^2 \theta$ is
\begin{flushright}
\small\textbf{Bihar STET TGT 2023}
\end{flushright}


\begin{choices}
\choice $\displaystyle 1$ 
\choice $\displaystyle \frac{1}{2}$ 
\choice $\displaystyle - \frac{1}{2}$ 
\choice $\displaystyle 0$ 
\end{choices}

\question 
If $\displaystyle \frac{ \cot 45^{\circ} }{\sin 30^{\circ} + \cos 60^{\circ} } $ is
\begin{flushright}
\small\textbf{TS Poly CET 2023}
\end{flushright}
\begin{choices}
\choice 2
\choice -2
\choice 1
\choice  -1
\end{choices}
\question The value of $\displaystyle \sqrt{\frac{1+ \sin \theta}{1-\sin \theta}}   $ is
\begin{flushright}
\small\textbf{TS Poly CET 2023}
\end{flushright}


\begin{choices}
\choice $\displaystyle \sec \theta + \tan \theta$
\choice $\displaystyle \cos \theta + \sin \theta$
\choice $\displaystyle \sec \theta + \cos \theta$ 
\choice $\displaystyle \sin \theta + \tan \theta$  
\end{choices}
\end{questions}
\end{multicols}

\begin{center}
\Large{Take the value of $\pi$ as 180${^\circ}$.}
\end{center}
%\end{document}

\newpage













\end{document}