\documentclass[11pt,paper=a4,answers]{exam}
% Allow the usage of UTF-8 characters
\usepackage[utf8]{inputenc}
% Allow the usage of graphics (.png, .jpg)


\usepackage{graphicx}
\usepackage{caption}
\usepackage{amsmath}
\usepackage{amsfonts}
\usepackage{amssymb}
\usepackage{multicol}
\usepackage{color}
%\usepackage{stix}
\usepackage{circuitikz}
\setlength{\columnseprule}{1pt}
\setlength{\columnsep}{1cm}
\def\columnseprulecolor{\color{black}}
\usepackage{blind text}
\usepackage{tikz}
%\usepackage{minipage}
\usepackage{lscape}
\usetikzlibrary{angles,quotes}
\usepackage{siunitx}
\usepackage[pdftex]{pict2e}
% Start the document
\begin{document}


\title{EAPCET - 2024 - QUESTION BANK}
\author{UNIC ACADEMY Team}
\maketitle
\newpage

\section*{18 May 2024 Forenoon}
\noindent \\
\rule{\textwidth}{1.4pt}


\subsection*{Mathematics}

\begin{multicols}{2}
\begin{questions}

% Question 1
\question
If a function $f: \mathbb{R} \rightarrow \mathbb{R}$ is defined by $f(x) = x^3 - x$, then $f$ is
\begin{flushright}
\small\textbf{AP EAPCET 2024}
\end{flushright}
\begin{choices}
  \choice one-one and onto
  \choice one-one but not onto
  \choice onto but not one-one
  \choice neither one-one nor onto
\end{choices}

% Question 2
\question
If $f(x) = \sqrt{x} - 1$ and $g(f(x)) = x + 2\sqrt{x} + 1$, then $g(x) =$ 
\begin{flushright}
\small\textbf{AP EAPCET 2024}
\end{flushright}
\begin{choices}
  \choice $(x + 2)^2$
  \choice $(x - 2)^2$
  \choice $(\sqrt{x} + 2)^2$
  \choice $(\sqrt{x} - 2)^2$
\end{choices}

% Question 3
\question
For all positive integers $n$, if $3\left(5^{2n+1}\right) + 2^{3n+1}$ is divisible by $k$, then the number of prime numbers less than or equal to $k$ is
\begin{flushright}
\small\textbf{AP EAPCET 2024}
\end{flushright}
\begin{choices}
  \choice $17$ 
  \choice $6$ 
  \choice $7$ 
  \choice $8$ 
\end{choices}


% Question 5
\question
If the determinant of a $3^{\text{rd}}$ order matrix $A$ is $K$, then the sum of the determinants of the matrices $(AA^T)$ and $(A-A^T)$ is
\begin{flushright}
\small\textbf{AP EAPCET 2024}
\end{flushright}
\begin{choices}
  \choice $2K$ 
  \choice $0$ 
  \choice $K^2$ 
  \choice $K$ 
\end{choices}


% Question 4
\question
If $\alpha, \beta, \gamma$ are the roots of \\ $\left|\begin{array}{ccc}
1-x & -2 & 1 \\
-2 & 4-x & -2 \\
1 & -2 & 1-x
\end{array}\right| = 0$, then \\ $\alpha \beta + \beta \gamma + \gamma \alpha =$
\begin{flushright}
\small\textbf{AP EAPCET 2024}
\end{flushright}
\begin{choices}
  \choice $6$ 
  \choice $8$ 
  \choice $0$ 
  \choice $-4$ 
\end{choices}


% Question 6
\question
While solving a system of linear equations $\mathrm{AX} = \mathrm{B}$ using Cramer's rule with the usual notation, if $\Delta = \left|\begin{array}{ccc}
1 & 1 & 1 \\
2 & -1 & 2 \\
-1 & 1 & 5
\end{array}\right|$, $\Delta_1 = \left|\begin{array}{ccc}
5 & 1 & 1 \\
4 & -1 & 2 \\
11 & 1 & 5
\end{array}\right|$, and $X = \begin{bmatrix} \alpha \\ 2 \\ \beta \end{bmatrix}$, then $\alpha^2 + \beta^2 =$
\begin{flushright}
\small\textbf{AP EAPCET 2024}
\end{flushright}
\begin{choices}
  \choice $9$ 
  \choice $13$ 
  \choice $5$ 
  \choice $25$ 
\end{choices}

% Question 7
\question
If the real parts of $\sqrt{-5-12i}$ and $\sqrt{5+12i}$ are positive values, the real part of $\sqrt{-8-6i}$ is a negative value, and $a + ib = \dfrac{\sqrt{-5-12i} + \sqrt{5+12i}}{\sqrt{-8-6i}}$, then $2a + b =$
\begin{flushright}
\small\textbf{AP EAPCET 2024}
\end{flushright}
\begin{choices}
  \choice $3$ 
  \choice $2$ 
  \choice $-3$ 
  \choice $-2$ 
\end{choices}





% Question 8
\question
The set of all real values of c for which the equation $z \bar{z}+(4-3 i) \bar{z}+(4+3 i) z+c=0$ represents a circle is
\begin{flushright}
\small\textbf{AP EAPCET 2024}
\end{flushright}
\begin{choices}
  \choice $[25, \infty)$
  \choice $[-5,5]$
  \choice $(-\infty,-5] \cup[5, \infty)$
  \choice $(-\infty, 25]$
\end{choices}



% Question 9
\question

If $Z=x+i y$ is a complex number, then the number of distinct solutions of the equation $z^3+\bar{z}=0$ is

\begin{flushright}
\small\textbf{AP EAPCET 2024}
\end{flushright}
\begin{choices}
  \choice $1$ 
  \choice $3$ 
  \choice Infinite
  \choice $5$ 
\end{choices}



% Question 10
\question
If the roots of the quadratic equation $x^2-35 x+c=0$ are in the ratio $2: 3$ and $\mathrm{c}=6 \mathrm{~K}$, then $\mathrm{K}=$
\begin{flushright}
\small\textbf{AP EAPCET 2024}
\end{flushright}
\begin{choices}
  \choice $49$ 
  \choice $14$ 
  \choice $21$ 
  \choice $7$ 
\end{choices}



% Question 11
\question
For real values of $x$ and $a$, if the expression $\dfrac{x+a}{2 x^2-3 x+1}$ assumes all real values, then
\begin{flushright}
\small\textbf{AP EAPCET 2024}
\end{flushright}
\begin{choices}
  \choice $a<-1$ or $a>-\dfrac{1}{2}$
  \choice $-1<a<-\dfrac{1}{2}$ 
  \choice  $\dfrac{1}{2}<a<1$
  \choice $a<\dfrac{1}{2}$ or $a>1$
\end{choices}



% Question 12
\question
If the sum of two roots $\alpha, \beta$ of the equation $x^4-x^3-8 x^2+2 x+12=0$ is zero and $\gamma, \delta(\gamma>\delta)$ are its other roots, then $3 \gamma+2 \delta=$
\begin{flushright}
\small\textbf{AP EAPCET 2024}
\end{flushright}
\begin{choices}
  \choice $0$ 
  \choice $1$ 
  \choice $3$ 
  \choice $5$ 
\end{choices}

% Question 13
\question
$f(x+h)=0$ represents the transformed equation of the equation
$f(x)=x^4+2 x^3-19 x^2-8 x+60=0$. If this transformation removes the term containing $x^3$ from $f(x)=0$, then $h=$
\begin{flushright}
\small\textbf{AP EAPCET 2024}
\end{flushright}
\begin{choices}
  \choice $-\dfrac{1}{2}$ 
  \choice $1$ 
  \choice $2$ 
  \choice $-1$ 
\end{choices}



% Question 14
\question
The number of different ways of preparing a garland by using 6 distinct white roses and 6 distinct red roses such that no two red roses come together is
\begin{flushright}
\small\textbf{AP EAPCET 2024}
\end{flushright}
\begin{choices}
  \choice $43200$ 
  \choice $86400$ 
  \choice $59200$ 
  \choice $76800$ 
\end{choices}

% Question 15
\question
The number of ways a committee of 8 members can be formed from a group of 10 men and 8 women such that the committee contains at most 5 men and at least 5 women is
\begin{flushright}
\small\textbf{AP EAPCET 2024}
\end{flushright}
\begin{choices}
  \choice $8061$ 
  \choice $8612$ 
  \choice $6082$ 
  \choice $8271$ 
\end{choices}

% Question 16
\question
If all the letters of the word CRICKET are permuted in all possible ways and the words (with or without meaning) thus formed are arranged in the dictionary order, then the rank of the word CRICKET is
\begin{flushright}
\small\textbf{AP EAPCET 2024}
\end{flushright}
\begin{choices}
  \choice $561$ 
  \choice $531$ 
  \choice $546$ 
  \choice $513$ 
\end{choices}


% Question 17
\question
The square root of independent term in the expansion of $\left(\dfrac{2 x^2}{5}+\sqrt{\dfrac{5}{x}}\right)^{10}$ is
\begin{flushright}
\small\textbf{AP EAPCET 2024}
\end{flushright}
\begin{choices}
  \choice $15 \sqrt{10}$
  \choice $10 \sqrt{15}$
  \choice $30 \sqrt{5}$
  \choice $20 \sqrt{5}$
\end{choices}


% Question 18
\question
The co-efficient of $x^5$ in $\left(3+x+x^2\right)^6$ is
\begin{flushright}
\small\textbf{AP EAPCET 2024}
\end{flushright}
\begin{choices}
  \choice $18$ 
  \choice $540$ 
  \choice $1620$ 
  \choice $2178$ 
\end{choices}

% Question 19
\question
The absolute value of the difference of the coefficients of $x^4$ and $x^6$ in the expansion of $\dfrac{2 x^2}{\left(x^2+1\right)\left(x^2+2\right)}$ is
\begin{flushright}
\small\textbf{AP EAPCET 2024}
\end{flushright}
\begin{choices}
  \choice $\dfrac{13}{4}$
  \choice $\dfrac{1}{4}$
  \choice $\dfrac{9}{4}$ 
  \choice $1$ 
\end{choices}


% Question 20
\question
$\tan 6^{\circ} \tan 42^{\circ} \tan 66^{\circ} \text { tan } 78^{\circ}=$
\begin{flushright}
\small\textbf{AP EAPCET 2024}
\end{flushright}
\begin{choices}
  \choice $\dfrac{3}{4}$
  \choice $1$ 
  \choice $0$ 
  \choice $\dfrac{1}{3}$ 
\end{choices}


% Question 21
\question
The maximum value of $12 \sin x-5 \cos x+3$ is
\begin{flushright}
\small\textbf{AP EAPCET 2024}
\end{flushright}
\begin{choices}
  \choice $18$ 
  \choice $13$ 
  \choice $16$ 
  \choice $10$ 
\end{choices}


% Question 22
\question
$\sin ^2 76^{\circ}+\sin ^2 16^{\circ}-\sin 76^{\circ} \sin 16^{\circ}=$
\begin{flushright}
\small\textbf{AP EAPCET 2024}
\end{flushright}
\begin{choices}
  \choice $0$ 
  \choice  $\dfrac{1}{4}$ 
  \choice  $\dfrac{3}{4}$
  \choice  $\dfrac{4}{3}$
\end{choices}


% Question 23
\question
$1+\sin x+\sin ^2 x+\sin ^3 x+\cdots+\infty=4+2 \sqrt{3}$ and $0<x<\pi, x \neq \dfrac{\pi}{2}$ then $x=$
\begin{flushright}
\small\textbf{AP EAPCET 2024}
\end{flushright}
\begin{choices}
  \choice $\dfrac{\pi}{6}, \dfrac{\pi}{4}$
  \choice $\dfrac{\pi}{4}, \dfrac{5 \pi}{6}$
  \choice $\dfrac{2 \pi}{5}, \dfrac{\pi}{6}$
  \choice $\dfrac{\pi}{3}, \dfrac{2 \pi}{3}$
\end{choices}



% Question 24
\question
$\operatorname{Tan}^{-1} 2+\operatorname{Tan}^{-1} 3=$
\begin{flushright}
\small\textbf{AP EAPCET 2024}
\end{flushright}
\begin{choices}
  \choice $-\dfrac{\pi}{4}$
  \choice $\dfrac{\pi}{4}$ 
  \choice $\dfrac{3 \pi}{4}$ 
  \choice $\dfrac{5 \pi}{4}$
\end{choices}


% Question 25
\question
$\operatorname{Cosh}^{-1} 2=$
\begin{flushright}
\small\textbf{AP EAPCET 2024}
\end{flushright}
\begin{choices}
  \choice $\log (2+\sqrt{3})$ 
  \choice $\log (2+\sqrt{5})$
  \choice $\log (2-\sqrt{5})$ 
  \choice $\log (2+\sqrt{2})$
\end{choices}



% Question 26
\question
In $\triangle A B C, \cos A+\cos B+\cos C=$
\begin{flushright}
\small\textbf{AP EAPCET 2024}
\end{flushright}
\begin{choices}
  \choice $1+\dfrac{r}{2 R}$ 
  \choice $1-\dfrac{r}{ R}$
  \choice $1+\dfrac{R}{r}$
  \choice $1+\dfrac{r}{ R}$
\end{choices}



% Question 27
\question
In a $\triangle A B C$ if $a=26, b=30, \cos c=\dfrac{63}{65}$ then $\mathrm{c}=$
\begin{flushright}
\small\textbf{AP EAPCET 2024}
\end{flushright}
\begin{choices}
  \choice $2$ 
  \choice $4$ 
  \choice $6$ 
  \choice $8$ 
\end{choices}


% Question 28
\question
If $H$ is orthocentre of $\triangle A B C$ and $A H=x ; B H=y ; C H=z$ then $\dfrac{a b c}{x y z}=$
\begin{flushright}
\small\textbf{AP EAPCET 2024}
\end{flushright}
\begin{choices}
  \choice $1$ 
  \choice $\dfrac{a+b+c}{x+y+z}$
  \choice $\dfrac{a}{x}+\dfrac{b}{y}+\dfrac{c}{z}$
  \choice $\dfrac{a b+b c+c a}{x y+y z+z x}$
\end{choices}



% Question 29
\question
In a regular hexagon $A B C D E F, \overline{A B}=\bar{a}$ and $\overline{B C}=\bar{b}$, then $\overline{F A}=$
\begin{flushright}
\small\textbf{AP EAPCET 2024}
\end{flushright}
\begin{choices}
  \choice $\bar{a}-\bar{b}$
  \choice $\bar{a}+\bar{b}$ 
  \choice $\bar{b}-\bar{a}$ 
  \choice $2 \bar{b}-\bar{a}$
\end{choices}


% Question 30
\question
If the points with position vectors $(\alpha \bar{i}+10 \bar{j}+13 \bar{k}),(6 \bar{i}+11 \bar{j}+11 \bar{k}),\left(\dfrac{9}{2} \bar{i}+\beta \bar{j}-8 \bar{k}\right)$ are collinear then $(19 \alpha-6 \beta)^2=$
\begin{flushright}
\small\textbf{AP EAPCET 2024}
\end{flushright}
\begin{choices}
  \choice $16$ 
  \choice $36$ 
  \choice $25$ 
  \choice $49$ 
\end{choices}



% Question 31
\question
If $\bar{f}, \bar{g}, \bar{h}$ be mutually orthogonal vectors of equal magnitudes, then the angle between the vectors $\bar{f}+\bar{g}+\bar{h}$ and $\bar{h}$ is
\begin{flushright}
\small\textbf{AP EAPCET 2024}
\end{flushright}
\begin{choices}
  \choice $\cos ^{-1}\left(\dfrac{\sqrt{3}}{4}\right)$
  \choice $\cos ^{-1}\left(\dfrac{1}{\sqrt{3}}\right)$
  \choice $\pi-\cos ^{-1}\left(\dfrac{1}{\sqrt{3}}\right)$
  \choice $\pi-\cos ^{-1}\left(\dfrac{\sqrt{3}}{4}\right)$
\end{choices}


% Question 32
\question
Let $\bar{a}, \bar{b}$ be two unit vector: If $\bar{c}=\bar{a}+2 \bar{b}$ and $\bar{d}=5 \bar{a}-4 \bar{b}$ are perpendicular to each other, then the angle between $\bar{a}$ and $\bar{b}$ is
\begin{flushright}
\small\textbf{AP EAPCET 2024}
\end{flushright}
\begin{choices}
  \choice $\dfrac{\pi}{6}$
  \choice $\dfrac{\pi}{4}$
  \choice $\dfrac{\pi}{3}$ 
  \choice $\dfrac{\pi}{8}$ 
\end{choices}

% Question 33
\question
If the vectors $\bar{a}=2 \bar{i}-\bar{j}+\bar{k}, \bar{b}=\bar{i}+2 \bar{j}-3 \bar{k}, \bar{c}=3 \bar{i}+p \bar{j}+5 \bar{k}$ are coplanar then $p=$
\begin{flushright}
\small\textbf{AP EAPCET 2024}
\end{flushright}
\begin{choices}
  \choice $4$ 
  \choice $14$ 
  \choice $-4$ 
  \choice $41$ 
\end{choices}



% Question 34
\question
For a set of observations, if the coefficient of variation is 25 and mean is 44 , then the variance is
\begin{flushright}
\small\textbf{AP EAPCET 2024}
\end{flushright}
\begin{choices}
  \choice $11$ 
  \choice $121$ 
  \choice $110$ 
  \choice $19$ 
\end{choices}


% Question 35
\question
If 5 letters are to be placed in 5 -addressed envelopes, then the probability that at least one letter is placed in the wrongly addressed envelope is
\begin{flushright}
\small\textbf{AP EAPCET 2024}
\end{flushright}
\begin{choices}
  \choice $\dfrac{1}{5}$
  \choice $\dfrac{1}{120}$
  \choice $\dfrac{4}{5}$
  \choice $\dfrac{119}{120}$
\end{choices}


% Question 36
\question
A student writes an examination which contains eight true or false questions. If he answers six or more questions correctly, he passes the examination. If the student answers all the questions, then the probability that he fails in the examination is
\begin{flushright}
\small\textbf{AP EAPCET 2024}
\end{flushright}
\begin{choices}
  \choice $\dfrac{37}{256}$
  \choice $\dfrac{19}{256}$
  \choice $\dfrac{119}{256}$
  \choice $\dfrac{219}{256}$
\end{choices}




% Question 37
\question
The probability that a person goes to college by car is $\dfrac{1}{5}$; by bus $\dfrac{2}{5}$ and by train is $\dfrac{3}{5}$ respectively. The probabilities that he reaches the college late if he takes car, bus, train are $\dfrac{2}{7}, \dfrac{4}{7}$ and $\dfrac{1}{7}$ respectively. If he reaches the college in time, the probability that he travelled by car is
\begin{flushright}
\small\textbf{AP EAPCET 2024}
\end{flushright}
\begin{choices}
  \choice $\dfrac{6}{29}$
  \choice $\dfrac{24}{29}$
  \choice $\dfrac{5}{29}$
  \choice $\dfrac{23}{29}$
\end{choices}



% Question 38
\question
$P, Q$ and $R$ try to hit the same target one after the other. If their probabilities of hitting the target are $\dfrac{2}{3}, \dfrac{3}{5}, \dfrac{5}{7}$ respectively, then the probability that the target is hit by $P$ or $Q$ but not by $R$ is
\begin{flushright}
\small\textbf{AP EAPCET 2024}
\end{flushright}
\begin{choices}
  \choice $\dfrac{26}{105}$
  \choice $\dfrac{79}{105}$
\choice $0$ 
  \choice $\dfrac{75}{105}$
\end{choices}


% Question 39
\question
A box contains $20 \%$ defective bulbs. Five bulbs are chosen randomly from this box. The probability that exactly 3 of the chosen bulbs are defective is
\begin{flushright}
\small\textbf{AP EAPCET 2024}
\end{flushright}
\begin{choices}
  \choice $\dfrac{32}{625}$
  \choice $\dfrac{32}{125}$
  \choice $\dfrac{16}{625}$
  \choice $\dfrac{16}{125}$
\end{choices}


% Question 40
\question
If a random variable $X$ satisfies poisson distribution with a mean value of 5 , then probability that $X<3$ is
\begin{flushright}
\small\textbf{AP EAPCET 2024}
\end{flushright}
\begin{choices}
  \choice $\dfrac{37}{2} e^5$
  \choice $6 e^5$ 
  \choice $6 e^{-5}$ 
  \choice $\dfrac{37}{2} e^{-5}$
\end{choices}





\end{questions}
\end{multicols}

\begin{center}
  \Large{Take the value of $\pi$ as 180${^\circ}$.}
\end{center}

\newpage



\subsection*{Physics}
\begin{multicols}{2}
\begin{questions}
\question
If $\tan  \theta = \sqrt{3}$, then the value of $\sec \theta$ is
\begin{flushright}
\small\textbf{AP Poly CET 2023}
\end{flushright}


\begin{choices}
\choice $\displaystyle 2$ 
\choice $\displaystyle \dfrac{1}{2}$ 
\choice $\displaystyle \dfrac{\sqrt{3}}{2}$ 
\choice $\displaystyle \dfrac{2}{\sqrt{3}}$  
\end{choices}
\end{questions}

\end{multicols}

\begin{center}
\Large{Take the value of $\pi$ as 180${^\circ}$.}
\end{center}
%\end{document}

\newpage















\subsection*{Chemistry}
\begin{multicols}{2}
\begin{questions}
% Question 1
\question
If $\tan  \theta = \sqrt{3}$, then the value of $\sec \theta$ is
\begin{flushright}
\small\textbf{AP Poly CET 2023}
\end{flushright}


\begin{choices}
\choice $\displaystyle 2$ 
\choice $\displaystyle \dfrac{1}{2}$ 
\choice $\displaystyle \dfrac{\sqrt{3}}{2}$ 
\choice $\displaystyle \dfrac{2}{\sqrt{3}}$  
\end{choices}
\end{questions}

\end{multicols}

\begin{center}
\Large{Take the value of $\pi$ as 180${^\circ}$.}
\end{center}
%\end{document}

\newpage













\end{document}