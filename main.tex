\documentclass[11pt,paper=a4,answers]{exam}
% Allow the usage of UTF-8 characters
\usepackage[utf8]{inputenc}
% Allow the usage of graphics (.png, .jpg)


\usepackage{graphicx}
\usepackage{caption}
\usepackage{amsmath}
\usepackage{amsfonts}
\usepackage{amssymb}
\usepackage{multicol}
\usepackage{color}
%\usepackage{stix}
\usepackage{circuitikz}
\setlength{\columnseprule}{1pt}
\setlength{\columnsep}{1cm}
\def\columnseprulecolor{\color{black}}
\usepackage{blind text}
\usepackage{tikz}
%\usepackage{minipage}
\usepackage{lscape}
\usetikzlibrary{angles,quotes}
\usepackage{siunitx}
\usepackage[pdftex]{pict2e}
% Start the document
\begin{document}


\title{EAPCET - 2024 - QUESTION BANK}
\author{UNIC ACADEMY Team}
\maketitle
\newpage

\section*{18 May 2024 Forenoon}
\noindent \\
\rule{\textwidth}{1.4pt}


\subsection*{Mathematics}

\begin{multicols}{2}
\begin{questions}

% Question 1
\question
If a function $f: \mathbb{R} \rightarrow \mathbb{R}$ is defined by $f(x) = x^3 - x$, then $f$ is
\begin{flushright}
\small\textbf{AP EAPCET 2024}
\end{flushright}
\begin{choices}
  \choice one-one and onto
  \choice one-one but not onto
  \choice onto but not one-one
  \choice neither one-one nor onto
\end{choices}

% Question 2
\question
If $f(x) = \sqrt{x} - 1$ and $g(f(x)) = x + 2\sqrt{x} + 1$, then $g(x) =$ 
\begin{flushright}
\small\textbf{AP EAPCET 2024}
\end{flushright}
\begin{choices}
  \choice $(x + 2)^2$
  \choice $(x - 2)^2$
  \choice $(\sqrt{x} + 2)^2$
  \choice $(\sqrt{x} - 2)^2$
\end{choices}

% Question 3
\question
For all positive integers $n$, if $3\left(5^{2n+1}\right) + 2^{3n+1}$ is divisible by $k$, then the number of prime numbers less than or equal to $k$ is
\begin{flushright}
\small\textbf{AP EAPCET 2024}
\end{flushright}
\begin{choices}
  \choice $17$ 
  \choice $6$ 
  \choice $7$ 
  \choice $8$ 
\end{choices}


% Question 5
\question
If the determinant of a $3^{\text{rd}}$ order matrix $A$ is $K$, then the sum of the determinants of the matrices $(AA^T)$ and $(A-A^T)$ is
\begin{flushright}
\small\textbf{AP EAPCET 2024}
\end{flushright}
\begin{choices}
  \choice $2K$ 
  \choice $0$ 
  \choice $K^2$ 
  \choice $K$ 
\end{choices}


% Question 4
\question
If $\alpha, \beta, \gamma$ are the roots of \\ $\left|\begin{array}{ccc}
1-x & -2 & 1 \\
-2 & 4-x & -2 \\
1 & -2 & 1-x
\end{array}\right| = 0$, then \\ $\alpha \beta + \beta \gamma + \gamma \alpha =$
\begin{flushright}
\small\textbf{AP EAPCET 2024}
\end{flushright}
\begin{choices}
  \choice $6$ 
  \choice $8$ 
  \choice $0$ 
  \choice $-4$ 
\end{choices}


% Question 6
\question
While solving a system of linear equations $\mathrm{AX} = \mathrm{B}$ using Cramer's rule with the usual notation, if $\Delta = \left|\begin{array}{ccc}
1 & 1 & 1 \\
2 & -1 & 2 \\
-1 & 1 & 5
\end{array}\right|$, $\Delta_1 = \left|\begin{array}{ccc}
5 & 1 & 1 \\
4 & -1 & 2 \\
11 & 1 & 5
\end{array}\right|$, and $X = \begin{bmatrix} \alpha \\ 2 \\ \beta \end{bmatrix}$, then $\alpha^2 + \beta^2 =$
\begin{flushright}
\small\textbf{AP EAPCET 2024}
\end{flushright}
\begin{choices}
  \choice $9$ 
  \choice $13$ 
  \choice $5$ 
  \choice $25$ 
\end{choices}

% Question 7
\question
If the real parts of $\sqrt{-5-12i}$ and $\sqrt{5+12i}$ are positive values, the real part of $\sqrt{-8-6i}$ is a negative value, and $a + ib = \dfrac{\sqrt{-5-12i} + \sqrt{5+12i}}{\sqrt{-8-6i}}$, then $2a + b =$
\begin{flushright}
\small\textbf{AP EAPCET 2024}
\end{flushright}
\begin{choices}
  \choice $3$ 
  \choice $2$ 
  \choice $-3$ 
  \choice $-2$ 
\end{choices}





% Question 8
\question
The set of all real values of c for which the equation $z \bar{z}+(4-3 i) \bar{z}+(4+3 i) z+c=0$ represents a circle is
\begin{flushright}
\small\textbf{AP EAPCET 2024}
\end{flushright}
\begin{choices}
  \choice $[25, \infty)$
  \choice $[-5,5]$
  \choice $(-\infty,-5] \cup[5, \infty)$
  \choice $(-\infty, 25]$
\end{choices}



% Question 9
\question

If $Z=x+i y$ is a complex number, then the number of distinct solutions of the equation $z^3+\bar{z}=0$ is

\begin{flushright}
\small\textbf{AP EAPCET 2024}
\end{flushright}
\begin{choices}
  \choice $1$ 
  \choice $3$ 
  \choice Infinite
  \choice $5$ 
\end{choices}



% Question 10
\question
If the roots of the quadratic equation $x^2-35 x+c=0$ are in the ratio $2: 3$ and $\mathrm{c}=6 \mathrm{~K}$, then $\mathrm{K}=$
\begin{flushright}
\small\textbf{AP EAPCET 2024}
\end{flushright}
\begin{choices}
  \choice $49$ 
  \choice $14$ 
  \choice $21$ 
  \choice $7$ 
\end{choices}



% Question 11
\question
For real values of $x$ and $a$, if the expression $\dfrac{x+a}{2 x^2-3 x+1}$ assumes all real values, then
\begin{flushright}
\small\textbf{AP EAPCET 2024}
\end{flushright}
\begin{choices}
  \choice $a<-1$ or $a>-\dfrac{1}{2}$
  \choice $-1<a<-\dfrac{1}{2}$ 
  \choice  $\dfrac{1}{2}<a<1$
  \choice $a<\dfrac{1}{2}$ or $a>1$
\end{choices}



% Question 12
\question
If the sum of two roots $\alpha, \beta$ of the equation $x^4-x^3-8 x^2+2 x+12=0$ is zero and $\gamma, \delta(\gamma>\delta)$ are its other roots, then $3 \gamma+2 \delta=$
\begin{flushright}
\small\textbf{AP EAPCET 2024}
\end{flushright}
\begin{choices}
  \choice $0$ 
  \choice $1$ 
  \choice $3$ 
  \choice $5$ 
\end{choices}

% Question 13
\question
$f(x+h)=0$ represents the transformed equation of the equation
$f(x)=x^4+2 x^3-19 x^2-8 x+60=0$. If this transformation removes the term containing $x^3$ from $f(x)=0$, then $h=$
\begin{flushright}
\small\textbf{AP EAPCET 2024}
\end{flushright}
\begin{choices}
  \choice $-\dfrac{1}{2}$ 
  \choice $1$ 
  \choice $2$ 
  \choice $-1$ 
\end{choices}



% Question 14
\question
The number of different ways of preparing a garland by using 6 distinct white roses and 6 distinct red roses such that no two red roses come together is
\begin{flushright}
\small\textbf{AP EAPCET 2024}
\end{flushright}
\begin{choices}
  \choice $43200$ 
  \choice $86400$ 
  \choice $59200$ 
  \choice $76800$ 
\end{choices}

% Question 15
\question
The number of ways a committee of 8 members can be formed from a group of 10 men and 8 women such that the committee contains at most 5 men and at least 5 women is
\begin{flushright}
\small\textbf{AP EAPCET 2024}
\end{flushright}
\begin{choices}
  \choice $8061$ 
  \choice $8612$ 
  \choice $6082$ 
  \choice $8271$ 
\end{choices}

% Question 16
\question
If all the letters of the word CRICKET are permuted in all possible ways and the words (with or without meaning) thus formed are arranged in the dictionary order, then the rank of the word CRICKET is
\begin{flushright}
\small\textbf{AP EAPCET 2024}
\end{flushright}
\begin{choices}
  \choice $561$ 
  \choice $531$ 
  \choice $546$ 
  \choice $513$ 
\end{choices}


% Question 17
\question
The square root of independent term in the expansion of $\left(\dfrac{2 x^2}{5}+\sqrt{\dfrac{5}{x}}\right)^{10}$ is
\begin{flushright}
\small\textbf{AP EAPCET 2024}
\end{flushright}
\begin{choices}
  \choice $15 \sqrt{10}$
  \choice $10 \sqrt{15}$
  \choice $30 \sqrt{5}$
  \choice $20 \sqrt{5}$
\end{choices}


% Question 18
\question
The co-efficient of $x^5$ in $\left(3+x+x^2\right)^6$ is
\begin{flushright}
\small\textbf{AP EAPCET 2024}
\end{flushright}
\begin{choices}
  \choice $18$ 
  \choice $540$ 
  \choice $1620$ 
  \choice $2178$ 
\end{choices}

% Question 19
\question
The absolute value of the difference of the coefficients of $x^4$ and $x^6$ in the expansion of $\dfrac{2 x^2}{\left(x^2+1\right)\left(x^2+2\right)}$ is
\begin{flushright}
\small\textbf{AP EAPCET 2024}
\end{flushright}
\begin{choices}
  \choice $\dfrac{13}{4}$
  \choice $\dfrac{1}{4}$
  \choice $\dfrac{9}{4}$ 
  \choice $1$ 
\end{choices}


% Question 20
\question
$\tan 6^{\circ} \tan 42^{\circ} \tan 66^{\circ} \text { tan } 78^{\circ}=$
\begin{flushright}
\small\textbf{AP EAPCET 2024}
\end{flushright}
\begin{choices}
  \choice $\dfrac{3}{4}$
  \choice $1$ 
  \choice $0$ 
  \choice $\dfrac{1}{3}$ 
\end{choices}


% Question 21
\question
The maximum value of $12 \sin x-5 \cos x+3$ is
\begin{flushright}
\small\textbf{AP EAPCET 2024}
\end{flushright}
\begin{choices}
  \choice $18$ 
  \choice $13$ 
  \choice $16$ 
  \choice $10$ 
\end{choices}


% Question 22
\question
$\sin ^2 76^{\circ}+\sin ^2 16^{\circ}-\sin 76^{\circ} \sin 16^{\circ}=$
\begin{flushright}
\small\textbf{AP EAPCET 2024}
\end{flushright}
\begin{choices}
  \choice $0$ 
  \choice  $\dfrac{1}{4}$ 
  \choice  $\dfrac{3}{4}$
  \choice  $\dfrac{4}{3}$
\end{choices}


% Question 23
\question
$1+\sin x+\sin ^2 x+\sin ^3 x+\cdots+\infty=4+2 \sqrt{3}$ and $0<x<\pi, x \neq \dfrac{\pi}{2}$ then $x=$
\begin{flushright}
\small\textbf{AP EAPCET 2024}
\end{flushright}
\begin{choices}
  \choice $\dfrac{\pi}{6}, \dfrac{\pi}{4}$
  \choice $\dfrac{\pi}{4}, \dfrac{5 \pi}{6}$
  \choice $\dfrac{2 \pi}{5}, \dfrac{\pi}{6}$
  \choice $\dfrac{\pi}{3}, \dfrac{2 \pi}{3}$
\end{choices}



% Question 24
\question
$\operatorname{Tan}^{-1} 2+\operatorname{Tan}^{-1} 3=$
\begin{flushright}
\small\textbf{AP EAPCET 2024}
\end{flushright}
\begin{choices}
  \choice $-\dfrac{\pi}{4}$
  \choice $\dfrac{\pi}{4}$ 
  \choice $\dfrac{3 \pi}{4}$ 
  \choice $\dfrac{5 \pi}{4}$
\end{choices}


% Question 25
\question
$\operatorname{Cosh}^{-1} 2=$
\begin{flushright}
\small\textbf{AP EAPCET 2024}
\end{flushright}
\begin{choices}
  \choice $\log (2+\sqrt{3})$ 
  \choice $\log (2+\sqrt{5})$
  \choice $\log (2-\sqrt{5})$ 
  \choice $\log (2+\sqrt{2})$
\end{choices}



% Question 26
\question
In $\triangle A B C, \cos A+\cos B+\cos C=$
\begin{flushright}
\small\textbf{AP EAPCET 2024}
\end{flushright}
\begin{choices}
  \choice $1+\dfrac{r}{2 R}$ 
  \choice $1-\dfrac{r}{ R}$
  \choice $1+\dfrac{R}{r}$
  \choice $1+\dfrac{r}{ R}$
\end{choices}



% Question 27
\question
In a $\triangle A B C$ if $a=26, b=30, \cos c=\dfrac{63}{65}$ then $\mathrm{c}=$
\begin{flushright}
\small\textbf{AP EAPCET 2024}
\end{flushright}
\begin{choices}
  \choice $2$ 
  \choice $4$ 
  \choice $6$ 
  \choice $8$ 
\end{choices}


% Question 28
\question
If $H$ is orthocentre of $\triangle A B C$ and $A H=x ; B H=y ; C H=z$ then $\dfrac{a b c}{x y z}=$
\begin{flushright}
\small\textbf{AP EAPCET 2024}
\end{flushright}
\begin{choices}
  \choice $1$ 
  \choice $\dfrac{a+b+c}{x+y+z}$
  \choice $\dfrac{a}{x}+\dfrac{b}{y}+\dfrac{c}{z}$
  \choice $\dfrac{a b+b c+c a}{x y+y z+z x}$
\end{choices}



% Question 29
\question
In a regular hexagon $A B C D E F, \overline{A B}=\bar{a}$ and $\overline{B C}=\bar{b}$, then $\overline{F A}=$
\begin{flushright}
\small\textbf{AP EAPCET 2024}
\end{flushright}
\begin{choices}
  \choice $\bar{a}-\bar{b}$
  \choice $\bar{a}+\bar{b}$ 
  \choice $\bar{b}-\bar{a}$ 
  \choice $2 \bar{b}-\bar{a}$
\end{choices}


% Question 30
\question
If the points with position vectors $(\alpha \bar{i}+10 \bar{j}+13 \bar{k}),(6 \bar{i}+11 \bar{j}+11 \bar{k}),\left(\dfrac{9}{2} \bar{i}+\beta \bar{j}-8 \bar{k}\right)$ are collinear then $(19 \alpha-6 \beta)^2=$
\begin{flushright}
\small\textbf{AP EAPCET 2024}
\end{flushright}
\begin{choices}
  \choice $16$ 
  \choice $36$ 
  \choice $25$ 
  \choice $49$ 
\end{choices}



% Question 31
\question
If $\bar{f}, \bar{g}, \bar{h}$ be mutually orthogonal vectors of equal magnitudes, then the angle between the vectors $\bar{f}+\bar{g}+\bar{h}$ and $\bar{h}$ is
\begin{flushright}
\small\textbf{AP EAPCET 2024}
\end{flushright}
\begin{choices}
  \choice $\cos ^{-1}\left(\dfrac{\sqrt{3}}{4}\right)$
  \choice $\cos ^{-1}\left(\dfrac{1}{\sqrt{3}}\right)$
  \choice $\pi-\cos ^{-1}\left(\dfrac{1}{\sqrt{3}}\right)$
  \choice $\pi-\cos ^{-1}\left(\dfrac{\sqrt{3}}{4}\right)$
\end{choices}


% Question 32
\question
Let $\bar{a}, \bar{b}$ be two unit vector: If $\bar{c}=\bar{a}+2 \bar{b}$ and $\bar{d}=5 \bar{a}-4 \bar{b}$ are perpendicular to each other, then the angle between $\bar{a}$ and $\bar{b}$ is
\begin{flushright}
\small\textbf{AP EAPCET 2024}
\end{flushright}
\begin{choices}
  \choice $\dfrac{\pi}{6}$
  \choice $\dfrac{\pi}{4}$
  \choice $\dfrac{\pi}{3}$ 
  \choice $\dfrac{\pi}{8}$ 
\end{choices}

% Question 33
\question
If the vectors $\bar{a}=2 \bar{i}-\bar{j}+\bar{k}, \bar{b}=\bar{i}+2 \bar{j}-3 \bar{k}, \bar{c}=3 \bar{i}+p \bar{j}+5 \bar{k}$ are coplanar then $p=$
\begin{flushright}
\small\textbf{AP EAPCET 2024}
\end{flushright}
\begin{choices}
  \choice $4$ 
  \choice $14$ 
  \choice $-4$ 
  \choice $41$ 
\end{choices}



% Question 34
\question
For a set of observations, if the coefficient of variation is 25 and mean is 44 , then the variance is
\begin{flushright}
\small\textbf{AP EAPCET 2024}
\end{flushright}
\begin{choices}
  \choice $11$ 
  \choice $121$ 
  \choice $110$ 
  \choice $19$ 
\end{choices}


% Question 35
\question
If 5 letters are to be placed in 5 -addressed envelopes, then the probability that at least one letter is placed in the wrongly addressed envelope is
\begin{flushright}
\small\textbf{AP EAPCET 2024}
\end{flushright}
\begin{choices}
  \choice $\dfrac{1}{5}$
  \choice $\dfrac{1}{120}$
  \choice $\dfrac{4}{5}$
  \choice $\dfrac{119}{120}$
\end{choices}


% Question 36
\question
A student writes an examination which contains eight true or false questions. If he answers six or more questions correctly, he passes the examination. If the student answers all the questions, then the probability that he fails in the examination is
\begin{flushright}
\small\textbf{AP EAPCET 2024}
\end{flushright}
\begin{choices}
  \choice $\dfrac{37}{256}$
  \choice $\dfrac{19}{256}$
  \choice $\dfrac{119}{256}$
  \choice $\dfrac{219}{256}$
\end{choices}




% Question 37
\question
The probability that a person goes to college by car is $\dfrac{1}{5}$; by bus $\dfrac{2}{5}$ and by train is $\dfrac{3}{5}$ respectively. The probabilities that he reaches the college late if he takes car, bus, train are $\dfrac{2}{7}, \dfrac{4}{7}$ and $\dfrac{1}{7}$ respectively. If he reaches the college in time, the probability that he travelled by car is
\begin{flushright}
\small\textbf{AP EAPCET 2024}
\end{flushright}
\begin{choices}
  \choice $\dfrac{6}{29}$
  \choice $\dfrac{24}{29}$
  \choice $\dfrac{5}{29}$
  \choice $\dfrac{23}{29}$
\end{choices}



% Question 38
\question
$P, Q$ and $R$ try to hit the same target one after the other. If their probabilities of hitting the target are $\dfrac{2}{3}, \dfrac{3}{5}, \dfrac{5}{7}$ respectively, then the probability that the target is hit by $P$ or $Q$ but not by $R$ is
\begin{flushright}
\small\textbf{AP EAPCET 2024}
\end{flushright}
\begin{choices}
  \choice $\dfrac{26}{105}$
  \choice $\dfrac{79}{105}$
\choice $0$ 
  \choice $\dfrac{75}{105}$
\end{choices}


% Question 39
\question
A box contains $20 \%$ defective bulbs. Five bulbs are chosen randomly from this box. The probability that exactly 3 of the chosen bulbs are defective is
\begin{flushright}
\small\textbf{AP EAPCET 2024}
\end{flushright}
\begin{choices}
  \choice $\dfrac{32}{625}$
  \choice $\dfrac{32}{125}$
  \choice $\dfrac{16}{625}$
  \choice $\dfrac{16}{125}$
\end{choices}


% Question 40
\question
If a random variable $X$ satisfies poisson distribution with a mean value of 5 , then probability that $X<3$ is
\begin{flushright}
\small\textbf{AP EAPCET 2024}
\end{flushright}
\begin{choices}
  \choice $\dfrac{37}{2} e^5$
  \choice $6 e^5$ 
  \choice $6 e^{-5}$ 
  \choice $\dfrac{37}{2} e^{-5}$
\end{choices}


% Question 41
\question
The equation $a x y+b y z=c y$ represent the locus of the points which lie on
\begin{flushright}
\small\textbf{AP EAPCET 2024}
\end{flushright}
\begin{choices}
  \choice zx -plane or on the planes perpendicular to zx -plane
  \choice on the planes perpendicular to x -axis
  \choice on the lines perpendicular to $\mathrm{z} x$-plane
  \choice on the lines perpendicular to $x y$-plane
\end{choices}

% Question 42
\question
If the axes are rotated through an angle $45^{\circ}$ about the origin in anticlockwise direction, then the transformed equation of $y^2=4 a x$ is
\begin{flushright}
\small\textbf{AP EAPCET 2024}
\end{flushright}
\begin{choices}
  \choice $(x+y)^2=4 \sqrt{2} a(x-y)$
  \choice $(x-y)^2=4 \sqrt{2} a(x+y)$
  \choice $(x-y)^2=\frac{4 a}{\sqrt{2}}(x+y)$
  \choice $(x+y)^2=\frac{4 a}{\sqrt{2}}(x-y)$
\end{choices}

% Question 43
\question
If the lines $3 x+y-4=0, x-\alpha y+10=0, \beta x+2 y+4=0$ and $3 x+y+k=0$ represent the sides of a square, then $\alpha \beta(k+4)^2=$
\begin{flushright}
\small\textbf{AP EAPCET 2024}
\end{flushright}
\begin{choices}
  \choice $-256$
  \choice $-512$ 
  \choice $-128$ 
  \choice $-1024$
\end{choices}

% Question 44
\question
A is the point of intersection of the lines $3 x+y-4=0$ and $x-y=0$. If a line having negative slope makes an angle of $45^{\circ}$ with the line $x-3 y+5=0$ and passes through A then its equation is
\begin{flushright}
\small\textbf{AP EAPCET 2024}
\end{flushright}
\begin{choices}
  \choice $x+y=2$
  \choice $x+2y=3$
  \choice $4x+3y=7$ 
  \choice $x+3y=4$
\end{choices}

% Question 45
\question
$2 x^2-3 x y-2 y^2=0$ represents two lines $L_1$ and $L_2 \cdot 2 x^2-3 x y-2 y^2-x+7 y-3=0$ represents another two lines $\mathrm{L}_3$ and $\mathrm{L}_4$. Let A be the point of intersection of lines $\mathrm{L}_1, \mathrm{~L}_3$ and B be the point of intersection of lines $L_2$ and $L_4$. The area of the triangle formed by lines $A B$ and $L_3, L_4$ is
\begin{flushright}
\small\textbf{AP EAPCET 2024}
\end{flushright}
\begin{choices}
  \choice $\frac{3}{10}$
  \choice $\frac{3}{5}$
  \choice $\frac{15}{2}$
  \choice $\frac{5}{2}$
\end{choices}

% Question 46
\question
The area of the triangle formed by the pair of lines $23 x^2-48 x y+3 y^2=0$ with the line $2 x+3 y+5=0$ is
\begin{flushright}
\small\textbf{AP EAPCET 2024}
\end{flushright}
\begin{choices}
  \choice $\frac{1}{13 \sqrt{3}} $
  \choice $\frac{25}{13 \sqrt{3}}$ 
  \choice $\frac{7}{13 \sqrt{5}}$ 
  \choice $\frac{9}{25 \sqrt{3}}$
\end{choices}

% Question 47
\question
If $\theta$ is the angle between the tangents drawn from the point $(2,3)$ to the circle $x^2+y^2-6 x+4 y+12=0$, then $\theta=$
\begin{flushright}
\small\textbf{AP EAPCET 2024}
\end{flushright}
\begin{choices}
  \choice $\operatorname{Cos}^{-1}\left(\frac{5}{13}\right)$
  \choice $\operatorname{Sin}^{-1}\left(\frac{4}{5}\right)$
  \choice $2 \operatorname{Tan}^{-1}\left(\frac{5}{12}\right)$
  \choice $\operatorname{Tan}^{-1}\left(\frac{5}{12}\right)$
\end{choices}

% Question 48
\question
If $2 x-3 y+3=0$ and $x+2 y+k=0$ are conjugate lines with respect to the circle $S \equiv x^2+y^2+8 x-6 y-24=0$, then the length of the tangent drawn from the point $\left(\frac{k}{4}, \frac{k}{3}\right)$ to the circle $S=0$ is
\begin{flushright}
\small\textbf{AP EAPCET 2024}
\end{flushright}
\begin{choices}
  \choice $7$
  \choice $1$ 
  \choice $12$ 
  \choice $24$
\end{choices}

% Question 49
\question
If $\mathrm{Q}(\mathrm{h}, \mathrm{k})$ is the inverse point of the point $\mathrm{P}(1,2)$ with respect to the circle $x^2+y^2-4 x+1=0$, then $2 \mathrm{~h}+\mathrm{k}=$
\begin{flushright}
\small\textbf{AP EAPCET 2024}
\end{flushright}
\begin{choices}
  \choice $3$
  \choice $4$ 
  \choice $7$ 
  \choice $11$
\end{choices}

% Question 50
\question
If ( $a, b$ ) and ( $c$, d) are the internal and external centres of similitudes of the circles $x^2+y^2+4 x-5=0$ and $x^2+y^2-6 y+8=0$ respectively, then $(a+d)(b+c)=$
\begin{flushright}
\small\textbf{AP EAPCET 2024}
\end{flushright}
\begin{choices}
  \choice $4$
  \choice $9$ 
  \choice $13$ 
  \choice $22$
\end{choices}

% Question 51
\question
A Circle S passes through the points of intersection of the circles $x^2+y^2-2 x+2 y-2=0$ and $x^2+y^2+2 x-2 y+1=0$. If the centre of this circle $S$ lies on the line $x-y+6=0$, then the radius of the circle $S$ is
\begin{flushright}
\small\textbf{AP EAPCET 2024}
\end{flushright}
\begin{choices}
  \choice $\sqrt{5}$
  \choice $5$ 
  \choice $\sqrt{41}$
  \choice $\sqrt{14}$
\end{choices}

% Question 52
\question
The line $x-2 y-3=0$ cuts the parabola $y^2=4 a x$ at the points P and Q . If the focus of this parabola is $\left(\frac{1}{4}, k\right)$, then $\mathrm{PQ}=$
\begin{flushright}
\small\textbf{AP EAPCET 2024}
\end{flushright}
\begin{choices}
  \choice $16 a \sqrt{5}$
  \choice $8 a \sqrt{5}$ 
  \choice $4 a \sqrt{5}$ 
  \choice $2 a \sqrt{5}$
\end{choices}

% Question 53
\question
If $4 x-3 y-5=0$ is a normal to the ellipse $3 x^2+8 y^2=k$, then the equation of the tangent drawn to this ellipse at the point $(-2, m)(m>0)$ is
\begin{flushright}
\small\textbf{AP EAPCET 2024}
\end{flushright}
\begin{choices}
  \choice $3 x+4 y-14=0$
  \choice $3 x-4 y+10=0$
  \choice $3 x-4 y+1=0$
  \choice $4 x+3 y-3=0$
\end{choices}

% Question 54
\question
If the line $5 x-2 y-6=0$ is a tangent to the hyperbola $5 x^2-k y^2=12$, then the equation of the normal to this hyperbola at the point $(\sqrt{6}, p)(p<0)$ is
\begin{flushright}
\small\textbf{AP EAPCET 2024}
\end{flushright}
\begin{choices}
  \choice $\sqrt{6} x+2 y=0$
  \choice $2\sqrt{6} x+3 y=3$ 
  \choice $\sqrt{6} x-5 y=21$ 
  \choice $3\sqrt{6} x- y=21$
\end{choices}

% Question 55
\question
If the angle between the asymptotes of the hyperbola $x^2-k y^2=3$ is $\frac{\pi}{3}$ and e is its eccentricity, then the pole of the line $x+y-1=0$ with respect to this hyperbola is
\begin{flushright}
\small\textbf{AP EAPCET 2024}
\end{flushright}
\begin{choices}
  \choice $\left(k, \frac{\sqrt{3} e}{2}\right)$
  \choice $\left(-k, \frac{\sqrt{3} e}{2}\right)$ 
  \choice $\left(-k,- \frac{\sqrt{3} e}{2}\right)$ 
  \choice $\left(k, - \frac{\sqrt{3} e}{2}\right)$
\end{choices}

% Question 56
\question
Let $P(\alpha, 4,7)$ and $Q(3, \beta, 8)$ are two points. If $Y Z$ - plane divides the join of the points $P$ and $Q$ in the ratio 2:3 and $Z X$ - plane divides the join of $P$ and $Q$ in the ratio $4: 5$, then length of line segment $P Q$ is
\begin{flushright}
\small\textbf{AP EAPCET 2024}
\end{flushright}
\begin{choices}
  \choice $\sqrt{107}$
  \choice $\sqrt{27}$ 
  \choice $\sqrt{83}$ 
  \choice $\sqrt{97}$
\end{choices}

% Question 57
\question
If $(\alpha, \beta, \gamma)$ are the Direction cosines of an angular bisector of two lines whose Direction ratios are $(2,2,1)$ and $(2,-1,-2)$, then $(\alpha+\beta+\gamma)^2=$
\begin{flushright}
\small\textbf{AP EAPCET 2024}
\end{flushright}
\begin{choices}
  \choice $3$
  \choice $2$ 
  \choice $4$ 
  \choice $5$
\end{choices}

% Question 58
\question
If the distance between the planes $2 x+y+z+1=0$ and $2 x+y+z+\alpha=0$ is 3 units, then product of all possible values of $\alpha$ is
\begin{flushright}
\small\textbf{AP EAPCET 2024}
\end{flushright}
\begin{choices}
  \choice $-43$
  \choice $43$ 
  \choice $53$ 
  \choice $-53$
\end{choices}

% Question 59
\question
$\lim _{x \rightarrow 0} \frac{1-\cos x \cos 2 x}{\sin ^2 x}=$
\begin{flushright}
\small\textbf{AP EAPCET 2024}
\end{flushright}
\begin{choices}
  \choice $\frac{11}{4}$
  \choice $\frac{5}{2}$ 
  \choice $3$ 
  \choice $5$
\end{choices}

% Question 60
\question
$\lim _{x \rightarrow \infty}\left(\frac{3 x^2-2 x+3}{3 x^2+x-2}\right)^{3 x-2}=$
\begin{flushright}
\small\textbf{AP EAPCET 2024}
\end{flushright}
\begin{choices}
  \choice $-3$
  \choice $e^{-1}$
  \choice $e^{-3}$
  \choice $-1$
\end{choices}

% Question 61
\question
$$
f(x)= \begin{cases}\frac{\left(2 x^2-a x+1\right)-\left(a x^2+3 b x+2\right)}{x+1} & , \text { if } x \neq-1 \\ k & , \text { if } x=-1\end{cases}
$$
is a real valued function. If $\mathrm{a}, \mathrm{b}, \mathrm{k} \in R$ and $f$ is continuous on R then $\mathrm{k}=$
\begin{flushright}
\small\textbf{AP EAPCET 2024}
\end{flushright}
\begin{choices}
  \choice $-\frac{1}{3}$
  \choice $6$ 
  \choice $\mathrm{a}-2$ 
  \choice $\mathrm{a}-3$
\end{choices}

% Question 62
\question
If $f(x)=\left\{\begin{array}{cc}\frac{2 x e^{1 / 2 x}-3 x e^{-1 / 2 x}}{e^{1 / 2 x}+4 e^{-1 / 2 x}} & \text { if } x \neq 0 \\ 0 & \text { if } x=0\end{array}\right.$ is a real valued function then
\begin{flushright}
\small\textbf{AP EAPCET 2024}
\end{flushright}
\begin{choices}
  \choice $f^{\prime}(0+)=\frac{-3}{4}$
  \choice $f^{\prime}(0-)=2$
  \choice $f$ is not differentiable at $x=0$
  \choice $f$ is differentiable at $x=0$
\end{choices}

% Question 63
\question
If $y=\operatorname{Tan}^{-1}\left(\frac{2-3 \sin x}{3-2 \sin x}\right)$ then $\frac{d y}{d x}=$
\begin{flushright}
\small\textbf{AP EAPCET 2024}
\end{flushright}
\begin{choices}
  \choice $\frac{(3-2 \sin x)^2}{13 \sin ^2 x-24 \sin x+13}$
  \choice $\frac{-5 \cos x}{13 \sin ^2 x-24 \sin x+13}$
  \choice $\frac{5 \sin \mathrm{x}}{13 \sin ^2 \mathrm{x}-24 \sin \mathrm{x}+13}$ 
  \choice $\frac{-5 \sin \mathrm{x}}{13 \sin ^2 \mathrm{x}-24 \sin \mathrm{x}+13}$
\end{choices}

% Question 64
\question
If $x=3\left[\sin t-\log \left(\cot \frac{t}{2}\right)\right]$ and $y=6\left[\cos t+\log \left(\tan \frac{t}{2}\right)\right]$ then $\frac{d y}{d x}=$
\begin{flushright}
\small\textbf{AP EAPCET 2024}
\end{flushright}
\begin{choices}
  \choice $\frac{2 \sin ^2 t}{1+\sin t \cos t}$
  \choice $\frac{2 \cos ^2 t}{1+\sin 2 t} $ 
  \choice $\frac{2 \cos ^2 t}{1+\sin t \cos t}$ 
  \choice $\frac{1+\cos 2 t}{1+\sin 2 t}$
\end{choices}





% Question 65
\question
By considering $1^{\prime}=0.0175$, the approximate value of $\cot 45^{\circ} 2^{\prime}$ is
\begin{flushright}
\small\textbf{AP EAPCET 2024}
\end{flushright}
\begin{choices}
  \choice $1.07$
  \choice $0.965$ 
  \choice $1.035$ 
  \choice $0.93$
\end{choices}

% Question 66
\question
A point is moving on the curve $y=x^3-3 x^2+2 x-1$ and the $y$-coordinate of the point is increasing at the rate of 6 units per second. When the point is at $(2,-1)$, the rate of change of x -coordinate of the point is
\begin{flushright}
\small\textbf{AP EAPCET 2024}
\end{flushright}
\begin{choices}
  \choice $3$
  \choice $\frac{1}{2}$
  \choice $-\frac{1}{2}$
  \choice $-3$
\end{choices}

% Question 67
\question
The length of the tangent drawn at the point $p\left(\frac{\pi}{4}\right)$ on the curve $x^{2 / 3}+y^{2 / 3}=2^{2 / 3}$ is
\begin{flushright}
\small\textbf{AP EAPCET 2024}
\end{flushright}
\begin{choices}
  \choice $\frac{2}{3}$
  \choice $1$ 
  \choice $\frac{4}{3}$
  \choice $2$
\end{choices}

1

2
% Question 68
\question
The set of all real values of ' $a$ ' such that the real valued function $f(x)=x^3+2 a x^2+3(a+1) x+5$ is strictly increasing in its entire domain is
\begin{flushright}
\small\textbf{AP EAPCET 2024}
\end{flushright}
\begin{choices}
  \choice $\left(-\infty,-\frac{3}{4}\right) \cup(3, \infty)$
  \choice $\left(-\frac{3}{4}, 3\right)$ 
  \choice $(1,3)$
  \choice $(-\infty, 1) \cup(3, \infty)$
\end{choices}

% Question 69
\question
$\int \frac{1}{x^5 \sqrt[5]{x^5+1}} d x=$
\begin{flushright}
\small\textbf{AP EAPCET 2024}
\end{flushright}
\begin{choices}
  \choice $\frac{4}{\sqrt[5]{x^5+1}}+c$
  \choice $4 x^4\left(x^5+1\right)^{4 / 5}+c $ 
  \choice $ -\frac{\left(x^5+1\right)^{4 / 5}}{4 x^4}+c$ 
  \choice $-\frac{\left(x^5+1\right)^{4 / 5}}{4 x^5}+c$
\end{choices}


% Question 70
\question
$\int \frac{x+1}{\sqrt{x^2+x+1}} \mathrm{~d} x=$
\begin{flushright}
\small\textbf{AP EAPCET 2024}
\end{flushright}
\begin{choices}
  \choice $\frac{1}{2} \sqrt{x^2+x+1}+\frac{1}{2} \cosh ^{-1}\left(\frac{x+2}{\sqrt{3}}\right)+\mathrm{c}$
  \choice $\frac{1}{2} \sqrt{x^2+x+1}+\frac{2}{\sqrt{3}} \operatorname{Tan}^{-1}\left(\frac{2 x+1}{\sqrt{3}}\right)+\mathrm{c}$ 
  \choice $\sqrt{x^2+x+1}+\frac{2}{\sqrt{3}} \log \left|x^2+x+1\right|+\mathrm{c}$ 
  \choice $\sqrt{x^2+x+1}+\frac{1}{2} \sinh ^{-1}\left(\frac{2 x+1}{\sqrt{3}}\right)+\mathrm{c}$
\end{choices}



% Question 71
\question
$\int\left(\tan ^7 x+\tan x\right) \mathrm{d} x=$
\begin{flushright}
\small\textbf{AP EAPCET 2024}
\end{flushright}
\begin{choices}
  \choice $\frac{\tan ^2 x}{12}\left(2 \tan ^4 x-3 \tan ^2 x+6\right)+c$
  \choice $\frac{\tan ^2 x}{6}-\frac{\tan ^5 x}{4}+\frac{\tan ^4 x}{2}+c$ 
  \choice $\frac{\tan ^2 x}{6}\left(\tan ^4 x+3 \tan ^2 x+4\right)+c$ 
  \choice $\frac{\tan x}{12}\left(\tan ^4 x-3 \tan ^2 x+6\right)+c$
\end{choices}



% Question 72
\question
$\int \frac{\operatorname{cosec} x}{3 \cos x+4 \sin x} \mathrm{~d} x=$
\begin{flushright}
\small\textbf{AP EAPCET 2024}
\end{flushright}
\begin{choices}
  \choice $\frac{1}{2} \log \left|\frac{\cos x}{3 \sin x+4 \cos x}\right|+\mathrm{c}$
  \choice $\frac{1}{3} \log \left|\frac{\sin x}{3 \cos x+4 \sin x}\right|+\mathrm{c}$ 
  \choice $\frac{1}{3} \log \left|\frac{3 \cos x+\sin x}{3 \cos x+4 \sin x}\right|+\mathrm{c} $ 
  \choice $\frac{1}{2} \log \left|\frac{\cos x+4 \sin x}{3 \cos x+4 \sin x}\right|+\mathrm{c}$
\end{choices}


% Question 73
\question
$\int e^{2 x+3} \sin 6 x \mathrm{~d} x=$
\begin{flushright}
\small\textbf{AP EAPCET 2024}
\end{flushright}
\begin{choices}
  \choice $\frac{e^{2 x+3}}{40}(2 \sin 6 x+6 \cos 6 x)+c$
  \choice $\frac{e^{2 x+3}}{40}(2 \cos 6 x+6 \sin 6 x)+c$ 
  \choice $\frac{e^{2 x+3}}{20}(\sin 6 x-3 \cos 6 x)+c$ 
  \choice $\frac{e^{2 x+3}}{20}(\cos 6 x-3 \sin 6 x)+c$
\end{choices}


% Question 74
\question
$\lim _{n \rightarrow \infty} n^4\left[\frac{1}{n^5}+\frac{1}{\left(n^2+1\right)^{\frac{5}{2}}}+\frac{1}{\left(n^2+4\right)^{\frac{5}{2}}}+\frac{1}{\left(n^2+9\right)^{\frac{5}{2}}}+\ldots+\frac{1}{4 \sqrt{2} n^5}\right]=$
\begin{flushright}
\small\textbf{AP EAPCET 2024}
\end{flushright}
\begin{choices}
  \choice $\frac{3}{4 \sqrt{2}}$
  \choice $\frac{3 \sqrt{2}}{4}$ 
  \choice $\frac{5}{6 \sqrt{2}}$ 
  \choice $\frac{5 \sqrt{2}}{6}$
\end{choices}

% Question 75
\question
$\int_{\log 4}^{\log 5} \frac{e^{2 x}+e^x}{e^{2 x}-5 e^x+6} d x=$
\begin{flushright}
\small\textbf{AP EAPCET 2024}
\end{flushright}
\begin{choices}
  \choice $ \log \left(\frac{64}{9}\right)$
  \choice $\log \left(\frac{256}{81}\right)$ 
  \choice $ \log \left(\frac{32}{3}\right)$ 
  \choice $\log \left(\frac{128}{27}\right)$
\end{choices}

$\begin{aligned} & \\ &  \\ & \\ & \end{aligned}$
% Question 76
\question
$\int_1^2 \frac{x^4-1}{x^6-1} d x=$
\begin{flushright}
\small\textbf{AP EAPCET 2024}
\end{flushright}
\begin{choices}
  \choice $\frac{1}{\sqrt{3}} \operatorname{Tan}^{-1}\left(\frac{\sqrt{3}}{2}\right)$
  \choice $\frac{121}{6} $ 
  \choice $\sqrt{2}-1$ 
  \choice $\frac{1}{\sqrt{2}} \operatorname{Tan}^{-1}\left(\frac{2}{\sqrt{3}}\right)$
\end{choices}

% Question 77
\question
The area of the region (in sq. units) enclosed by the curve $y=x^3-19 x+30$ and the X -axis is
\begin{flushright}
\small\textbf{AP EAPCET 2024}
\end{flushright}
\begin{choices}
  \choice $\frac{167}{2}$ 
  \choice $\frac{517}{2}$ 
  \choice $36$ 
  \choice $72$
\end{choices}

% Question 78
\question
The differential equation representing the family of circles having their centres on Y -axis is $\left(y_1=\frac{d y}{d x}\right.$ and $\left.y_2=\frac{d^2 y}{d x^2}\right)$
\begin{flushright}
\small\textbf{AP EAPCET 2024}
\end{flushright}
\begin{choices}
  \choice $y_2=y\left(y_1{ }^2+1\right)$
  \choice $y_2=x y\left(y_1{ }^2+1\right) $ 
  \choice $ x y_2=y_1\left(y_1^2+1\right)$ 
  \choice $ x y_2=y\left(y_1^2+1\right)$
\end{choices}

% Question 79
\question
The general solution of the differential equation $\left(\sin y \cos ^2 y-x \sec ^2 y\right) d y=(\tan y) d x$ is
\begin{flushright}
\small\textbf{AP EAPCET 2024}
\end{flushright}
\begin{choices}
  \choice $\tan y=3 x \cos ^3 y+c $
  \choice $x(\sec y+\tan y)=\cos ^2 y+c$ 
  \choice $ y \sin y=x^2 \cos ^2 y+c$ 
  \choice $3 x \tan y+\cos ^3 y=c$
\end{choices}

% Question 80
\question
The general solution of the differential equation $(x-y-1) d y=(x+y+1) d x$ is
\begin{flushright}
\small\textbf{AP EAPCET 2024}
\end{flushright}
\begin{choices}
  \choice $\operatorname{Tan}^{-1}\left(\frac{y+1}{x}\right)-\frac{1}{2} \log \left(x^2+y^2+2 y+1\right)=c$
  \choice $(x-y)+\log (x+y)=c$ 
  \choice $y^2-x^2+x y-3 y-x=c $ 
  \choice $(x-y-1)^2(x+y+1)^3=c$
\end{choices}


\end{questions}
\end{multicols}

\begin{center}
  \Large{Take the value of $\pi$ as 180${^\circ}$.}
\end{center}

\newpage



\subsection*{Physics}
\begin{multicols}{2}
\begin{questions}
\question
If $\tan  \theta = \sqrt{3}$, then the value of $\sec \theta$ is
\begin{flushright}
\small\textbf{AP Poly CET 2023}
\end{flushright}


\begin{choices}
\choice $\displaystyle 2$ 
\choice $\displaystyle \dfrac{1}{2}$ 
\choice $\displaystyle \dfrac{\sqrt{3}}{2}$ 
\choice $\displaystyle \dfrac{2}{\sqrt{3}}$  
\end{choices}
\end{questions}

\end{multicols}

\begin{center}
\Large{Take the value of $\pi$ as 180${^\circ}$.}
\end{center}
%\end{document}

\newpage















\subsection*{Chemistry}
\begin{multicols}{2}
\begin{questions}
% Question 1
\question
If $\tan  \theta = \sqrt{3}$, then the value of $\sec \theta$ is
\begin{flushright}
\small\textbf{AP Poly CET 2023}
\end{flushright}


\begin{choices}
\choice $\displaystyle 2$ 
\choice $\displaystyle \dfrac{1}{2}$ 
\choice $\displaystyle \dfrac{\sqrt{3}}{2}$ 
\choice $\displaystyle \dfrac{2}{\sqrt{3}}$  
\end{choices}
\end{questions}

\end{multicols}

\begin{center}
\Large{Take the value of $\pi$ as 180${^\circ}$.}
\end{center}
%\end{document}

\newpage













\end{document}